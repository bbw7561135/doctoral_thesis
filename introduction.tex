\documentclass[../thesis.tex]{subfiles}
\begin{document}
\chapter{Introduction}
\section{From the Big Bang to Now}
How did we get here?  
This is one of the most fundamental questions humanity asks itself.  
At its most basic, this is also the question the field of cosmology attempts to answer: what is the origin of the Universe, how has it evolved over time, and what is its ultimate fate? 
Indeed, the past century has seen remarkable progress in this line of inquiry.
Beginning with Einstein's publication of the general theory of relativity \citep{Einstein1916}, the past 100 years have seen our understanding of the Universe and our place in it revolutionized. 
In the mid-1920s we realized that the Universe is in fact much larger than just our Milky Way Galaxy \citep{Hubble1925}, and the diffuse nebulae observed by astronomers were actually (previously incomprehensibly) distant galaxies similar to our own.
Starting with Hubble's 1929 discovery that the Universe is expanding \citep{Hubble1929}, and finally confirmed by Penzias and Wilson's detection of the Cosmic Microwave Background \citep[CMB;][]{PenziasWilson1965, Dickeetal1965}, a strong body of evidence has led us to the conclusion that the Universe began in a Big Bang and has been expanding ever since.
Other observations have since revealed that the overwhelming majority of the mass in the Universe is actually composed of invisible dark matter \citep{RubinThonnardFord1980}, and the normal baryonic matter comprising all stars, planets, and every living thing accounts for only about a fifth of the total mass budget. 
In fact, normal matter constitutes an even smaller 4\% of the total energy content of the Universe, with the majority consisting of dark energy, the phenomenon responsible for the observed acceleration of the Universe's expansion \citep{Riessetal1998, Perlmutteretal1999}.
Most recently the advent of precision cosmology, led by high resolution studies of the CMB, has allowed us to determine the age, composition, and shape of the Universe with multiple-digit precision.  
This has led to the conclusion that we live in a flat universe which is 13.8 billion years old and composed primarily of dark energy and dark matter \citep{PlanckParams2015}.

While studies of the CMB have revealed much about the Universe on large scales, the CMB also provides a detailed picture of what the Universe looked like when atoms were first forming, $\sim$400,000 years after the Big Bang. 
However, the Universe thus revealed had little in common with the Universe we observe today.  
At these early times, matter and radiation were distributed quite evenly, and deviations from perfect uniformity were only present on the level of one part in 100,000---a far cry from the complex, highly structured tapestry of planets, stars, and galaxies we see today.
As even our deepest current observations reveal a Universe which is already highly structured, precisely how this transition occurred is one of the primary frontiers of modern cosmology.

When the next generation of astronomical facilities---including the \textit{Square Kilometer Array} (SKA), \textit{James Webb Space Telescope} (JWST), and 30-meter class ground-based telescopes like the \textit{Giant Magellan Telescope} (GMT)--- come online, they will allow us to peer much farther through space and time, opening even earlier epochs to observation.
The JWST in particular is expected to reveal the formation of the first galaxies at $z\simeq10$, and a detailed understanding of how these structures form will be crucial to interpreting the findings of these next-generation observatories.
Coupled with ever more sophisticated simulations enabled by advances in both technology and technique, we are rapidly approaching completion of a continuous timeline of cosmic history. 
By connecting \textit{ab-initio}, state-of-the-art cosmological simulations of structure formation to next-generation observations we will for the first time be able to construct a complete picture of cosmic evolution, stretching all the way from the simple initial conditions of the early universe to the present epoch.

\section{Cosmic Renaissance}
The basic process by which the first stars and galaxies form has been thoroughly established within the standard $\Lambda$CDM model for hierarchical structure formation.
Following the release of the CMB, cosmic evolution during the `dark ages' was relatively simple, with small perturbations in the density of cold dark matter 
growing slowly under the influence of gravity.
These over-densities eventually coalesce into gravitationally bound `minihalos' within which the first stars formed roughly 100 million years after the Big Bang, initiating a cosmic `renaissance' and marking a significant increase in the complexity of cosmic evolution \citep{BarkanaLoeb2001, Miralda-Escude2003, Brommetal2009, Loeb2010}.
The light from these stars swept through the Universe, beginning the process of reionization \citep[e.g.,][]{Meiksin2009}, while the heavy elements forged in their cores and released in the violent supernova explosions marking their deaths began the process of chemical enrichment \citep[e.g.,][]{KarlssonBrommHawthorn2013}.
Since structure in our $\Lambda$CDM Universe forms in a hierarchical manner, such that the dark matter halos hosting the first galaxies are built up from smaller progenitors which hosted the first stars, Pop III stars have a direct impact on the formation and evolution of all subsequent stages of galaxy formation, and thus are thus key drivers of cosmic evolution \citep{Bromm2013}.  

The complex baryonic physics involved have so far prevented a definitive answer to this question, but the basic properties of these Pop III stars have been reasonably well established, with the consensus that the first stars formed in dark matter minihalos on the order of $10^5-10^6$ solar masses at redshifts $z \simeq 20-30$ \citep{CouchmanRees1986,   HaimanThoulLoeb1996, Tegmarketal1997}.  
In the absence of any feedback, pioneering numerical work on the collapse of primordial gas into these halos, where molecular hydrogen is the only available coolant, suggested that the first stars were predominantly very massive, on the order of $100$\msun with a top-heavy initial mass function (IMF) (e.g., \citealt{BrommCoppiLarson1999, BrommCoppiLarson2002, AbelBryanNorman2002, BrommLarson2004, Yoshidaetal2006, OSheaNorman2007}).  
More recent work, aided by increased resolution, has revised this picture, finding that significant fragmentation occurs during the star formation process, resulting in lower characteristic masses and a much broader IMF than previously thought \citep{StacyGreifBromm2010,Clarketal2011a,Clarketal2011b,Greifetal2011,Greifetal2012,StacyBromm2013,Hiranoetal2014,Hosokawaetal2015}. This has led to an emerging consensus that the Pop III initial mass function (IMF) was somewhat top-heavy with a characteristic mass of $\sim$ a few $\times 10\msun$ \citep{Bromm2013}. 

While the formation of the very first stars is set by cosmological initial conditions, the formation of all subsequent generations is much more complex due to the feedback processes initiated by that very first episode of primordial star formation.
The characteristic mass of the first stars is therefore crucial, as it largely determines the influence Pop III stars have on their environment, controlling both their total luminosity and ionizing radiation output \citep{Schaerer2002}, as well as the details of their eventual demise \citep{Hegeretal2003,HegerWoosley2010,MaederMeynet2012}, thus setting the initial conditions for the formation of the first galaxies.

\section{The First Supernovae}
One crucial element in any realistic model of galaxy formation is supernova (SN) feedback, both through direct radiative and mechanical feedback from the explosion itself, and via the chemical enrichment process. 
Supernova feedback is of particular importance in the early universe, as the first stars tended to be rather more massive than those formed today. 
In this thesis, we aim to better understand the role these violent explosions play in the formation of the first galaxies, particularly through radiative feedback.

\textbf{Chapter 2} concerns the prospects for directly detecting the first supernovae.  Stars in the mass range $140-260\msun$ are expected to end their lives as extremely energetic Pair-Instability Supernovae (PISNe), which would be within the detection limits of the JWST even at $z=30$.
We tackle this question using a semi-analytic model to estimate the formation rate of minihalos hosting the first stars, coupled with a feedback prescription informed by simulations from \citet{PawlikMilosavljevicBromm2011} to determine what fraction of these minihalos likely hosted a PISN.
Given their brightness, the source density of these events is the limiting factor in their detection, even in the most optimistic scenario. 
We show that the optimal search strategy for finding a Pop III PISN is a mosaic-style search of many moderately deep exposures, as even rather high-redshift PISNe are unlikely to be missed, and a large area will need to be searched to ensure a detection.
While scientific consensus on the characteristic mass of Pop III stars---and thus, the PISN rate---has shifted in recent years, PISNe will still occur, albeit at a lower rate. 

\textbf{Chapters 3 and 4} focus on the impact radiative feedback from the first supernovae has on star formation in neighboring minihalos.  
As these events occur in a predominantly neutral medium, the vast majority of their ionizing output---both from the explosion itself, and the compact remnants left behind---is absorbed by the intergalactic medium (IGM). 
Consequently, most Pop III SN feedback is locally confined, similar to chemical feedback.  However, the neutral hydrogen cross-section for X-rays and cosmic rays is small, allowing them to escape their host minihalos to build up an isotropic X-ray (Chapter 3) or cosmic ray (Chapter 4) background.
This ionizing background radiation is then absorbed by high-density gas in neighboring minihalos, where it serves to enhance the ionization fraction.  
As free electrons are required to catalyze the formation of molecular hydrogen, which is the only available coolant in primordial gas, the presence of an ionizing background actually enhances the cooling efficiency of gas collapsing in pristine minihalos where no chemical enrichment has yet occurred.
We show that for both moderate X-ray and cosmic ray backgrounds, this effect overwhelms the associated heating, resulting in a net cooling of the collapsing gas and expediting collapse to high densities.  
This effect is particularly pronounced for cosmic rays, as they cause very little heating per ionization event.
The characteristic mass of the stars formed in the presence of such a background however appears to be quite robust, remaining stable even as the strength of the irradiating background varies by several orders of magnitude.

The primary tool employed throughout these investigations has been the numerical simulation of cosmological structure formation, specifically using the smoothed-particle hydrodynamics (SPH) simulation code \textsc{gadget-2} \citep{Springel2005}.  
Understanding the results of these simulations required a great deal of exploratory analysis, and in \textbf{chapter 5} we describe the novel software tools developed to enable this investigation, which have been released as the open-source GAdget DataFrame Library, or \code{gadfly}. The package is designed to leverage the capabilities of the broader python scientific computing ecosystem, and provides a framework for analyzing particle-based \textsc{gadget} and \textsc{gizmo} \citep{Hopkins2015} simulation data.  \code{Gadfly} enables  efficient memory management, includes utilities for unit handling, coordinate transformations, and parallel batch processing, and provides highly optimized routines for visualizing smoothed-particle hydrodynamics (SPH) datasets.
\end{document}