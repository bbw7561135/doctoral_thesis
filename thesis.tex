% !TeX program = pdflatex

\documentclass{utthesis}

\usepackage{lipsum} % Dummy text for examples
\usepackage{graphicx}
\usepackage[space]{grffile}
\usepackage{latexsym}
\usepackage{textcomp}
\usepackage{longtable}
\usepackage{multirow,booktabs}
\usepackage{amsfonts,amsmath,amssymb}
\usepackage{natbib}
\usepackage{url}
\usepackage[utf8]{inputenc}
\usepackage[english]{babel}
\usepackage[hidelinks,pagebackref]{hyperref} % Handy for links, can comment for plain PDF

\newcommand{\msun}{\,\mathrm{M}_{\odot}}


\UTyear=2016

\makeindex

\graphicspath{{./figures/}}

\begin{document}

\author{Jacob Alexander Hummel} 
\title{The First Supernovae: Impact on Early Star Formation and Prospects for Direct Detection} 
\date{Revised: \today}

\UTcopyrightlegend % Optional

\begin{UTcommittee}
\UTaddsupervisor{Volker Bromm}
\UTaddcommittee{J. Craig Wheeler}
\UTaddcommittee{Milo\v s Milosavljevi\'c}
\UTaddcommittee{Steven Finkelstein}
\UTaddcommittee{Naoki Yoshida}
\end{UTcommittee}

\UTtitlepage{B.S., M.A.}{May}

\frontmatter

\setcounter{page}{4}

%\include{acknowledgements}
%\include{preface}

\begin{UTabstract}{Volker Bromm}
The formation of the first stars in the Universe marked a pivotal moment in cosmic history, initiating the transition from the simple initial conditions of the big bang to the complex structures we see today.
Ionizing radiation produced by these so-called Population III stars began the process of reionization, and the supernovae marking their deaths initiated the process of chemical enrichment. 
These violent explosions and the compact remnants they leave behind also produce significant amounts of high-energy X-rays and cosmic rays able  to travel through the predominantly neutral intergalactic medium and build up a cosmic background.
To better understand how these first supernovae impact subsequent episodes of metal-free star formation, we employ \textit{ab-initio}, cosmological hydrodynamics simulations to model the formation of stars in a $\sim$10$^6\msun$ minihalo at $z \gtrsim 20$ under the influence of both an X-ray and cosmic ray background.
The presence of an ionizing background---whether X-rays or cosmic rays---serves to expedite the collapse of gas to high densities by enhancing molecular hydrogen cooling, thus allowing stars to form at substantially earlier epochs in strongly irradiated minihalos.
The mass of the stars thus formed however appears to be quite robust, maintaining a characteristic mass of order $\sim$ a few $\times10\msun$ even as the strength of the ionizing background varies by several orders of magnitude.
We furthermore assess the prospects for direct detection of the first supernovae should they happen to end their lives as extremely energetic pair-instability supernovae, which should be within the detection limits of the upcoming \textit{James Webb Space Telescope}.
Using a combination of semi-analytic models and cosmological simulations to estimate their source density, we find that the primary obstacle to observing such events is their scarcity, not their faintness.
Finally, we describe the novel software developed to enable this research.  These tools for manipulating and analyzing simulation data have been released as the open-source \textsc{gadget} DataFrame Library: \verb|gadfly|.
\end{UTabstract}


% add . to include section numbers (non-standard)
\tableofcontents
%\tableofcontents.

\listoffigures

\mainmatter

\chapter{Start}
\lipsum[3]
\section{One}
\subsection{Mini}
\section{Two}
\subsection{Miny}
\subsubsection{Mo}
\section{Three}
\section{This is very very very very very very long to test the guideline for wrapping in the table of contents}

\chapter{More}
\lipsum[3]
\begin{figure}
This is a figure.
\caption[This]{This is a caption.}
\end{figure}

\chapter{Finishing}
\lipsum[3]

\begin{appendix}

\chapter{First}
\lipsum[3]

\chapter{Second}
\section{Part one}
\lipsum[3]
\section{Part two}
\begin{figure}
This is a figure.
\caption[That]{This is also a caption.}
\end{figure}

\end{appendix}

\backmatter

\printindex

\cleardoublepage
\phantomsection
\addcontentsline{toc}{chapter}{Bibliography}
\chapter*{Bibliography}
\lipsum[2]

\end{document}
