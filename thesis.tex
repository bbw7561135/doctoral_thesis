% !TeX program = pdflatex

\documentclass[12pt,double]{thesis}
\usepackage{subfiles}
\usepackage[stable]{footmisc}
\usepackage{graphicx}
\usepackage{xspace}
\usepackage[space]{grffile}
\usepackage{latexsym}
\usepackage{textcomp}
\usepackage{longtable}
\usepackage{multirow,booktabs}
\usepackage{amsfonts,amsmath,amssymb}
\usepackage{natbib}
\usepackage{url}
\usepackage[utf8]{inputenc}
\usepackage[english]{babel}
\usepackage[hidelinks]{hyperref} % Handy for links, can comment for plain PDF
%%%%%%%%%%%%%%%%%%%%%%%
%%% Commands
%%%%%%%%%%%%%%%%%%%%%%%
%Units
\newcommand{\kelvin}{\ensuremath{\,\mathrm{K}}\xspace}
\newcommand{\s}{\ensuremath{\,\mathrm{s}}\xspace}
\newcommand{\grams}{\ensuremath{\,\mathrm{g}}\xspace}
\newcommand{\cm}{\ensuremath{\,\mathrm{cm}}\xspace}
\newcommand{\cc}{\ensuremath{\,\mathrm{cm}^{-3}}\xspace}
\newcommand{\msun}{\ensuremath{\,\mathrm{M}_{\odot}}\xspace}
\newcommand{\rsun}{\ensuremath{\,\mathrm{R}_{\odot}}\xspace}
\newcommand{\zsun}{\ensuremath{\,\mathrm{Z}_{\odot}}\xspace}
\newcommand{\kms}{\ensuremath{\,\mathrm{km}\,\mathrm{s}^{-1}\xspace}}
\newcommand{\kpc}{\ensuremath{\,\mathrm{kpc}}\xspace}
\newcommand{\pc}{\ensuremath{\,\mathrm{pc}}\xspace}
\newcommand{\au}{\ensuremath{\,\mathrm{AU}}\xspace}
\newcommand{\Mpc}{\ensuremath{\,\mathrm{Mpc}}\xspace}
\newcommand{\yrs}{\ensuremath{\,\mathrm{yrs}}\xspace}
\newcommand{\yr}{\ensuremath{\,\mathrm{yr}}\xspace}
\newcommand{\myr}{\ensuremath{\,\mathrm{Myr}}\xspace}
\newcommand{\ev}{\ensuremath{\,\mathrm{eV}}\xspace}
\newcommand{\kev}{\ensuremath{\,\mathrm{keV}}\xspace}
\newcommand{\erg}{\ensuremath{\,\mathrm{erg}}\xspace}
\newcommand{\sr}{\ensuremath{\,\mathrm{sr}}\xspace}
\newcommand{\Hz}{\ensuremath{\,\mathrm{Hz}}\xspace}
%Constants
\newcommand{\kb}{k_{\mathrm{B}}}
\newcommand{\mh}{m_{\mathrm{H}}}
%Special values
\newcommand{\tcmb}{\ensuremath{{T}_{\mathrm{CMB}}}\xspace}
\newcommand{\tff}{\ensuremath{{t}_{\mathrm{ff}}}\xspace}
\newcommand{\tH}{\ensuremath{{t}_{\textsc{h}}}\xspace}
\newcommand{\zcrit}{\ensuremath{{Z}_{\mathrm{crit}}}\xspace}
\newcommand{\ncrit}{\ensuremath{{n}_\mathrm{crit}}\xspace}
\newcommand{\tvir}{\ensuremath{{T}_{\mathrm{vir}}}\xspace}
\newcommand{\rvir}{\ensuremath{{R}_{\mathrm{vir}}}\xspace}
\newcommand{\mvir}{\ensuremath{{M}_{\mathrm{vir}}}\xspace}
\newcommand{\rta}{\ensuremath{{R}_{\mathrm{ta}}}\xspace}
%Chemistry
\newcommand{\htwo}{\ensuremath{\mathrm{H}_2}\xspace}
\newcommand{\hd}{\ensuremath{\mathrm{HD}}\xspace}
\newcommand{\deut}{\ensuremath{\mathrm{D}}\xspace}
\newcommand{\h}{\ensuremath{\mathrm{H}}\xspace}
\newcommand{\hplus}{\ensuremath{\mathrm{H}^+}\xspace}
\newcommand{\hminus}{\ensuremath{\mathrm{H}^-}\xspace}
\newcommand{\he}{\ensuremath{\mathrm{He}}\xspace}
\newcommand{\heplus}{\ensuremath{\mathrm{He}^+}\xspace}
\newcommand{\heminus}{\ensuremath{\mathrm{He}^-}\xspace}
\newcommand{\HI}{\ensuremath{\mathrm{H\,\textsc{i}}}\xspace}
\newcommand{\HeI}{\ensuremath{\mathrm{He\,\textsc{i}}}\xspace}
\newcommand{\HeII}{\ensuremath{\mathrm{He\,\textsc{ii}}}\xspace}
\newcommand{\abunde}{\ensuremath{ x_{\mathrm{e}}}\xspace}
\newcommand{\abundhe}{\ensuremath{ x_{\mathrm{He}}}\xspace}
\newcommand{\abundd}{\ensuremath{ x_{\mathrm{D}}}\xspace}
\newcommand{\abundhd}{\ensuremath{ x_{\mathrm{HD}}}\xspace}
\newcommand{\abundhtwo}{\ensuremath{ x_{\mathrm{H}_2} }\xspace}
\newcommand{\abundhplus}{\ensuremath{ x_{\mathrm{H}^+}}\xspace}
\newcommand{\abundi}{\ensuremath{ x_i}\xspace}
%Energetics
\newcommand{\etot}{\ensuremath{{E}_{\mathrm{tot}}}\xspace}
\newcommand{\egrav}{\ensuremath{{E}_{\mathrm{grav}}}\xspace}
\newcommand{\etherm}{\ensuremath{{E}_{\mathrm{th}}}\xspace}
\newcommand{\ekin}{\ensuremath{{E}_{\mathrm{kin}}}\xspace}
\newcommand{\erot}{\ensuremath{{E}_{\mathrm{rot}}}\xspace}
\newcommand{\ucr}{\ensuremath{{u}_{\textsc{cr}}}\xspace}
\newcommand{\ucrz}{\ensuremath{{u}_{\textsc{cr}}(z)}\xspace}
\newcommand{\uxr}{\ensuremath{{u}_{\textsc{xr}}}\xspace}
\newcommand{\uxrz}{\ensuremath{{u}_{\textsc{xr}}(z)}\xspace}
\newcommand{\gxr}{\ensuremath{\Gamma_{\textsc{xr}}}\xspace}
\newcommand{\gcrit}{\ensuremath{\Gamma_{\textsc{xr}, \mathrm{crit}}}\xspace}
\newcommand{\gblowout}{\ensuremath{\Gamma_{\textsc{xr}, \mathrm{blowout}}}\xspace}
%Radiation
\newcommand{\lya}{\mathrm{Ly}\alpha}
\newcommand{\jlw}{J_{\mathrm{LW},21}}
\newcommand{\Lxr}{\ensuremath{L_{\textsc{xr}}}\xspace}
\newcommand{\jxr}{\ensuremath{J_{\textsc{xr}}}\xspace}
\newcommand{\jxrz}{\ensuremath{J_{\textsc{xr}}(z)}\xspace}
\newcommand{\jxrvz}{\ensuremath{J_{\nu, \textsc{xr}}(z)}\xspace}
\newcommand{\jcrit}{\ensuremath{J_{\textsc{xr}, \mathrm{crit}}}\xspace}
\newcommand{\jblowout}{\ensuremath{J_{\textsc{xr}, \mathrm{blowout}}}\xspace}
%Cooling
\newcommand{\tcool}{t_{\mathrm{cool}}}
\newcommand{\tcoolhtwo}{t_{\mathrm{cool,H}_2}}
\newcommand{\tcoolhd}{t_{\mathrm{cool,HD}}}
%Cosmology
\newcommand{\Dhubble}{D_{\mathrm{H}}}
\newcommand{\omegab}{\Omega_{\mathrm{b}}}
\newcommand{\omegal}{\Omega_{\Lambda}}
\newcommand{\omegam}{\Omega_{\mathrm{m}}}
\newcommand{\omegatot}{\mathbf\Omega_{\mathrm{tot}}}
% Star Formation
\newcommand{\sfr}{\ensuremath{\Psi_{*}}\xspace}
\newcommand{\sfrz}{\ensuremath{\Psi_{*}(z)}\xspace}
\newcommand{\xrb}{\ensuremath{\textsc{hmxb}}\xspace}
%Computational
\newcommand{\rsoft}{r_{\mathrm{soft}}}
\newcommand{\code}[1]{\texttt{#1}}
%Mathematics
\newcommand{\curl}{\vec{\nabla}\times}  %curl, nabla times something
\newcommand{\dive}{\vec{\nabla}\cdot}   %divergence, nabla dot something
\newcommand{\vect}[1]{\boldsymbol{#1}}
%Latex 
\newcommand{\about}{\ensuremath{\sim}}
\newcommand{\RefTab}[1]{\mbox{Table~\ref{#1}}}                     
\newcommand{\RefFig}[1]{\mbox{Figure~\ref{#1}}}                    
\newcommand{\RefEq}[1]{\mbox{Eq.~(\ref{#1})}}                 
\newcommand{\RefCh}[1]{\mbox{Chapter~\ref{#1}}}                  
\newcommand{\RefSec}[1]{\mbox{Section~\ref{#1}}}                        
\newcommand{\RefApp}[1]{\mbox{Appendix~\ref{#1}}}
%%%%%%%%%%%%%%%%%%%%%%%
%%%%%%%%%%%%%%%%%%%%%%%
%%%%%%%%%%%%%%%%%%%%%%


\UTyear=2016

\makeindex

\graphicspath{{./figures/}}

\settocdepth{subsubsection}
\begin{document}

\author{Jacob Alexander Hummel} 
\title{Prospects for Directly Detecting the First Supernovae, and their Impact on Early Star Formation} 
\date{Revised: \today}

\UTcopyrightlegend % Optional

\begin{UTcommittee}
\UTaddsupervisor{Volker Bromm}
\UTaddcommittee{J. Craig Wheeler}
\UTaddcommittee{Milo\v s Milosavljevi\'c}
\UTaddcommittee{Steven Finkelstein}
\UTaddcommittee{Naoki Yoshida}
\end{UTcommittee}

\UTtitlepage{B.S.; M.A.}{May}

\frontmatter

\begin{UTdedication}
%For my parents, who taught me ever to question,
For Sarah, 

my love and my rock.

\phantom{.}

You are the one truly fixed point in my universe.
\end{UTdedication}

\cleardoublepage
\setcounter{page}{5}

\begin{UTacknowledgements}
This thesis marks the culmination of seven years of blood, sweat, and tears---all in the literal sense at one point or another. Seven years of sleepless nights and way too much coffee. Seven years of buggy code and results that made no sense. Seven years of self-doubt, frustration, mental and emotional exhaustion, and far too many eighteen-hour days. There have been many obstacles along the way---some external, some self-imposed---but given the opportunity, I'd do it again in a heartbeat.

Because it has also been seven years of laughter and joy. Of friendship, camaraderie, and \textit{gemütlichkeit}. Seven years of learning and self-discovery.  Seven years filled with moments of elation when the veil of confusion suddenly lifted and an idea \textit{clicked}. Seven years of pushing myself to the limit and discovering that at the end of the day, I was up to the challenge. Even the roadblocks along the way led to skills, both tangible and not, that made me into the person I am today, someone I am proud to be.

The science herein would not have been possible without the pioneering work of all those who came before me, first and foremost my advisor, Volker Bromm. 
It has been a unique honor to work with him and continue to drive this field forward.  
His guidance has left an indelible mark on both this work and who I am today.  
I have also been privileged with an exceptional committee of mentors, past and present. Milo\v s Milosavljevi\'c, Craig Wheeler, Steven Finkelstein, Karl Gebhardt, Eiichiro Komatsu, Naoki Yoshida, Ian Lindevald, Matt Beaky, and Vayujeet Gokhale: thank you for teaching me what it means to be a scientist both by example and through your imparted wisdom.  This work has also benefited greatly from a number of brilliant minds with whom I have had the opportunity to collaborate. Of special note: Andreas Pawlik, who guided me in the transition from wide-eyed undergraduate to scientific researcher, and was in many ways my de-facto advisor for the duration of my Master's thesis; Athena Stacy, who mentored me in the art of numerical simulations, and Chalence Safranek-Shrader, my friend and officemate throughout our shared tenure in Austin, and my favorite external sanity check.

On a personal note, I can think of no better way to have spent the bulk of my twenties, and that is in large part due to the amazing community I found here in Austin.  I owe my sanity to the incredible friends I've made over these past seven years, and to the Fridays at the Crown \& Anchor, the cookouts, the float trips, the barbecue pilgrimages, the games of volleyball, softball, basketball, and ultimate frisbee, the boat parties, and the weirdnights that we shared.  There are too many to name them all, but special honors go to John, Chalence, Chris, Kevin, Taylor, Keaton, Matt, Joel, Randi, Amanda, Tom, Paul, Manos, Eric, Brian, Rodolfo, Emma, Myoungwon, and Raquel. Thank you.

I would never have made it to this point without the love and support of my entire family, for which I am eternally grateful.  To my parents John and Catherine, thank you for always encouraging me to follow my dreams and supporting me at every step along the way, for teaching me ever to question, and for instilling in me the tenacity to follow through on the dreams of a seven year old boy.  To my siblings Raymond, Andrea, David, and Valerie, thank you for keeping me grounded, keeping me honest, and being the sort of people I can trust to always have my back, no matter what.  I don't know how many people have family willing to drive eighteen hours over the Rockies in the dead of winter just so they don't have to spend the holidays alone, but I'm glad I do. I've also been lucky enough to gain new family along the way.  Theresa, Tom, Shawn, Erin, Jackie, Maciek, Christy, Fritz, Renie, and all the rest, thank you for embracing me so thoroughly and completely.

Finally, and most importantly, this certainly would not have been possible without the unwavering love and support of my amazing wife, Sarah. She has stood by me without fail, despite having to spend more time in different states than the same room over the past seven years, believing in me when I couldn't believe in myself, and generally being the best life partner I could possibly have hoped for.  Thank you Sarah. For everything.

\begin{flushright}
\textsc{Jacob Alexander Hummel}
\end{flushright}

\begin{flushleft}
\textit{The University of Texas at Austin}

\textit{April 2016}
\end{flushleft}


\end{UTacknowledgements}

\begin{UTabstract}{Volker Bromm}
The formation of the first stars in the Universe marked a pivotal moment in cosmic history, initiating the transition from the simple initial conditions of the big bang to the complex structures we see today.
Ionizing radiation produced by these so-called Population III stars began the process of reionization, and the supernovae marking their deaths initiated the process of chemical enrichment. 
These violent explosions and the compact remnants they leave behind also produce significant amounts of high-energy X-rays and cosmic rays able  to travel through the predominantly neutral intergalactic medium and build up a cosmic background.
To better understand how these first supernovae impact subsequent episodes of metal-free star formation, we employ \textit{ab-initio}, cosmological hydrodynamics simulations to model the formation of stars in a $\sim$10$^6\msun$ minihalo at $z \gtrsim 20$ under the influence of both an X-ray and cosmic ray background.
The presence of an ionizing background---whether X-rays or cosmic rays---serves to expedite the collapse of gas to high densities by enhancing molecular hydrogen cooling, thus allowing stars to form at substantially earlier epochs in strongly irradiated minihalos.
The mass of the stars thus formed however appears to be quite robust, maintaining a characteristic mass of order $\sim$ a few $\times10\msun$ even as the strength of the ionizing background varies by several orders of magnitude.
We furthermore assess the prospects for direct detection of the first supernovae should they happen to end their lives as extremely energetic pair-instability supernovae, which should be within the detection limits of the upcoming \textit{James Webb Space Telescope}.
Using a combination of semi-analytic models and cosmological simulations to estimate their source density, we find that the primary obstacle to observing such events is their scarcity, not their faintness.
Finally, we describe the novel software developed to enable this research.  These tools for manipulating and analyzing simulation data have been released as the open-source \textsc{gadget} DataFrame Library: \verb|gadfly|.
\end{UTabstract}


% add . to include section numbers (non-standard)
%\tableofcontents
\tableofcontents.

\listoftables
\listoffigures

\mainmatter

\subfile{introduction}
\subfile{firstSNe}
\subfile{appendix}
\subfile{xrays}
\subfile{cosmic_rays}
\subfile{gadfly}
\subfile{outlook}

\backmatter

\printindex

\cleardoublepage
\phantomsection
\bibliographystyle{aasjournal}
\bibliography{biblio}

\end{document}
