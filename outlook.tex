\documentclass[../thesis.tex]{subfiles}
\begin{document}
\chapter{Outlook}
The upcoming launch of the \textit{James Webb Space Telescope} (JWST) heralds an exciting new era of astronomy. 
For the first time, we will be able to construct a complete timeline of cosmic history from the Big Bang to the present day.
For over a decade, astronomers have been hard at work filling the gap in our knowledge between the end of the dark ages and our deepest observations.
High-redshift galaxies have now been observed up to $z\sim8$, and the JWST is expected to reveal systems with virial masses $M_{\mathrm{vir}} = 10^9\msun$ at $z\simeq10$, just 500 million years after the Big Bang.
While not there just yet, we are rapidly acquiring the ability to explain how these galaxies formed from first principles by using sophisticated numerical simulation techniques to derive their expected observable properties.
While advances in supercomputing technology have yielded impressive dividends on this front in recent years simply by enhancing the maximum feasible resolution of our simulations, computing power alone will not solve this problem.
\textit{Ab-initio} cosmological simulations are still currently limited by computational resources, and improved understanding of the physical processes occurring on scales beyond the resolution limit of our most advanced state-of-the-art simulations will be crucial in deciphering the observations made by the JWST.

Accomplishing this task will require a thorough understanding of how the first stars influence their environment.
Ionizing radiation, metals, dust, and $\htwo$-dissociating Lyman-Werner radiation all dramatically increase the complexity of cosmic evolution, and feedback from the first supernovae has a hand in all of these.
If a significant fraction of the first stars explode as PISNe, the formation of the next generation could be delayed for up to $10^7$ years, a significant fraction of the Hubble time at such redshifts \citep{JohnsonGreifBromm2007, Yoshidaetal2007, Whalenetal2008}.
In addition, detection of such events remains our best opportunity for obtaining direct evidence of the Pop III initial mass function, though there is the possibility that clusters of Pop III stars forming under optimal conditions could be marginally detectable \citep{Safranek-Shraderetal2012}.

Independent of their mass, however, the first stars will doubtless introduce additional complexity and significantly influence all subsequent generations of star formation, thus directly impacting the observational signatures of the first galaxies.
Regardless of whether they die as PISNe or more common core-collapse supernovae, these violent explosions will release the first metals in the Universe, by extension allowing dust to form for the first time.  
Given the low critical metallicity threshold expected for the transition to Population II star formation, even a small number of Pop III supernovae can have an outsized impact on subsequent star formation, modulo efficient mixing of the chemically enriched gas \citep{Ritteretal2012}.

Radiation from the first stars heats and ionizes the gas surrounding them, potentially suppressing further star formation until the halo exceeds $\sim$$10^8\msun$, at which point gas can cool via atomic line transitions.  
Pop III stars live and die in a predominantly neutral medium however, such that the majority of the ionizing radiation they produce is absorbed by the intergalactic medium (IGM), and is unable to influence star formation in other minihalos.  
Supernovae and the compact remnants they leave behind, on the other hand, produce significant amounts of X-ray radiation and cosmic rays.
Since the neutral hydrogen cross-section for both of these is small, they can pass through the predominantly neutral IGM to build up a cosmic background.
Such an ionizing background has the net effect of enhancing the cooling efficiency of gas in pristine minihalos, as the elevated free electron fraction boosts the formation efficiency of $\htwo$, the only available coolant.
This effectively expedites collapse, as the gas in these halos is then able to fulfill the Rees-Ostriker criterion for collapse at an earlier epoch than it otherwise would.
This is particularly interesting given the highly biased nature of Pop III star formation, as stars forming in one minihalo can effectively trigger star formation in neighboring minihalos.

Intriguingly, the characteristic mass of the stars formed under X-ray or cosmic ray feedback appears to be quite stable under a wide range of ionizing backgrounds.
The thermodynamic state of the gas at high densities is quite robust, and compared to the canonical behavior in the absence of any irradiation, neither X-rays nor cosmic rays are able to significantly alter the behavior of the gas as it collapses to high densities.
This provides further support for the growing consensus that the the characteristic mass of Pop III is closer to $\sim$ a few $\times10\msun$ rather than $\sim100\msun$, and limits the possibility of forming low-mass Pop III stars which might have survived to the present.

The era of the first stars and galaxies is one of the cutting-edge frontiers of modern astrophysics, and the advent of next-generation space- and ground-based telescopes in the near future will continue to drive the field forward, necessitating increasingly sophisticated, large scale numerical simulations. 
The work presented in this dissertation represents one of many such small steps towards this common goal as we rapidly approach completion of a continuous picture of cosmic evolution from the Big Bang to the present day.

\end{document}